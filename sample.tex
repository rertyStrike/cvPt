%%%%%%%%%%%%%%%%%
% This is an sample CV template created using altacv.cls
% (v1.1.5, 1 December 2018) written by LianTze Lim (liantze@gmail.com). Now compiles with pdfLaTeX, XeLaTeX and LuaLaTeX.
%
%% It may be distributed and/or modified under the
%% conditions of the LaTeX Project Public License, either version 1.3
%% of this license or (at your option) any later version.
%% The latest version of this license is in
%%    http://www.latex-project.org/lppl.txt
%% and version 1.3 or later is part of all distributions of LaTeX
%% version 2003/12/01 or later.
%%%%%%%%%%%%%%%%

%% If you need to pass whatever options to xcolor
\PassOptionsToPackage{dvipsnames}{xcolor}

%% If you are using \orcid or academicons
%% icons, make sure you have the academicons
%% option here, and compile with XeLaTeX
%% or LuaLaTeX.
% \documentclass[10pt,a4paper,academicons]{altacv}

%% Use the "normalphoto" option if you want a normal photo instead of cropped to a circle
% \documentclass[10pt,a4paper,normalphoto]{altacv}

\documentclass[10pt,a4paper,ragged2e]{altacv}

%% AltaCV uses the fontawesome and academicon fonts
%% and packages.
%% See texdoc.net/pkg/fontawecome and http://texdoc.net/pkg/academicons for full list of symbols. You MUST compile with XeLaTeX or LuaLaTeX if you want to use academicons.

% Change the page layout if you need to
\geometry{left=1cm,right=9cm,marginparwidth=6.8cm,marginparsep=1.2cm,top=1.25cm,bottom=1.25cm}

% Change the font if you want to, depending on whether
% you're using pdflatex or xelatex/lualatex
\ifxetexorluatex
  % If using xelatex or lualatex:
  \setmainfont{Lato}
\else
  % If using pdflatex:
  \usepackage[utf8]{inputenc}
  \usepackage[T1]{fontenc}
  \usepackage[default]{lato}
\fi

% Change the colours if you want to
\definecolor{Mulberry}{HTML}{72243D}
\definecolor{SlateGrey}{HTML}{2E2E2E}
\definecolor{LightGrey}{HTML}{666666}
\colorlet{heading}{Sepia}
\colorlet{accent}{Mulberry}
\colorlet{emphasis}{SlateGrey}
\colorlet{body}{LightGrey}

% Change the bullets for itemize and rating marker
% for \cvskill if you want to
\renewcommand{\itemmarker}{{\small\textbullet}}
\renewcommand{\ratingmarker}{\faCircle}

%% sample.bib contains your publications
\addbibresource{sample.bib}

\begin{document}
\name{Victor Afonso dos Reis}
\tagline{}
\photo{2.5cm}{vdosreis}
\personalinfo{%
  % Not all of these are required!
  % You can add your own with \printinfo{symbol}{detail}
  \email{victor.afonsoreis35@gmail.com}
  \phone{+55 (17)982204427}
  \mailaddress{Av Belvedere, 750, QD A Lote 16, 15056030}
  \location{São José do Rio Preto-SP, Brasil}
  %\homepage{}
 %\twitter{}
  \linkedin{www.linkedin.com/in/vdosreis}
  %\github{}
  %% You MUST add the academicons option to \documentclass, then compile with LuaLaTeX or XeLaTeX, if you want to use \orcid or other academicons commands.
  % \orcid{orcid.org/0000-0000-0000-0000}
}

%% Make the header extend all the way to the right, if you want.
\begin{fullwidth}
\makecvheader
\end{fullwidth}

%% Depending on your tastes, you may want to make fonts of itemize environments slightly smaller
% \AtBeginEnvironment{itemize}{\small}

%% Provide the file name containing the sidebar contents as an optional parameter to \cvsection.
%% You can always just use \marginpar{...} if you do
%% not need to align the top of the contents to any
%% \cvsection title in the "main" bar.
\cvsection[sample-p1sidebar]{Experiência Profissional}

\cvevent{Estagiário em Engenharia de Telecomunicações}{Qualcomm}{Abr 2019 - Dez 2019}{São Paulo, Brasil}
\begin{itemize}
\item Análise de protocolos e testes de dispositivos 4G e 5G.
\end{itemize}

\divider

\cvevent{Estagiário em Engenharia}{Intel Corporation}{Fev 2018 -Fev 2019}{Munique, Alemanha}
\begin{itemize}
\item Integração, verificação e soluções de hardware para chips de modems G.Fast, VDSL e ADSL (Home Connected Division).
\item Desenvolver e adaptar scripts de testes para automação de testes baseados em Matlab.
\item Uso de equipamentos de medição, teste e modificação como: spectrum analyser, osciloscópios, loop simulators e estações de solda.
\end{itemize}

\cvsection{Experiência Acadêmica}

\cvevent{Bolsista de Iniação Científica}{Fundunesp/UNESP}{Jan 2016 - Dez 2018}{}
Estudo, desenvolvimento e implementação das codificações 8b/10b e 64b/66b. Modelou-se as codificações no Matlab (SIMULINK) e posteriormente implementou-se em um FPGA (Xilinx Kintex 7) em VHDL. 

\divider

\cvevent{Colaborador de Pesquisa}{São Paulo Research and Analysis Center (SPRACE)/UNESP}{Out 2015 - Dez 2018}{}
Estudante/Pesquisador na área de instrumentação eletrônica para física de altas energias. 

\divider

\cvevent{Voluntário}{PET Elétrica/UNESP}{Dez 2014 - Mar 2017}{}
A principal atividade desenvolvida no grupo foi a Oficina de Projetos na qual fui líder da atividade. Nesta atividade era desenvolvido projetos eletrônicos junto com os calouros do curso. 


\medskip

%\cvsection{A Day of My Life}

% Adapted from @Jake's answer from http://tex.stackexchange.com/a/82729/226
% \wheelchart{outer radius}{inner radius}{
% comma-separated list of value/text width/color/detail}
%\wheelchart{1.5cm}{0.5cm}{%
%  6/8em/accent!30/{Sleep,\\beautiful sleep},
%  3/8em/accent!40/Hopeful novelist by night,
%  8/8em/accent!60/Daytime job,
%  2/10em/accent/Sports and relaxation,
%  5/6em/accent!20/Spending time with family
%}

%\clearpage
%\cvsection[page2sidebar]{Publicações}
\cvsection{Publicações}

\nocite{*}

\printbibliography[heading=pubtype,title={\printinfo{\faBook}{Books}},type=book]

%\divider

\printbibliography[heading=pubtype,title={\printinfo{\faFileTextO}{Artigos}}, type=article]
Modeling and Implementation in FPGA of 8b/10b Encoding. SIIM/SPS. Nov 2017. Disponível em: <www.eventos.ufabc.edu.br/siimsps/files/id14.pdf>

\divider

Robustness Analysis and State Machine Modeling of 8b/10b Encoding. ERMAC. Mai 2017. Disponível em: <www.fc.unesp.br/Home/Departamentos/Matematica/ermac/caderno\-ermac\_2017.pdf>
%\divider

\printbibliography[heading=pubtype,title={\printinfo{\faGroup}{Conference Proceedings}},type=inproceedings]

%% If the NEXT page doesn't start with a \cvsection but you'd
%% still like to add a sidebar, then use this command on THIS
%% page to add it. The optional argument lets you pull up the
%% sidebar a bit so that it looks aligned with the top of the
%% main column.
% \addnextpagesidebar[-1ex]{page3sidebar}


\end{document}
